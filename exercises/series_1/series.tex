\documentclass[a4paper]{scrreprt}

% Uncomment to optimize for double-sided printing.
% \KOMAoptions{twoside}

% Set binding correction manually, if known.
% \KOMAoptions{BCOR=2cm}

% Localization options
\usepackage[english]{babel}
\usepackage[T1]{fontenc}
\usepackage[utf8]{inputenc}

% Quotations
\usepackage{dirtytalk}

% Enhanced verbatim sections. We're mainly interested in
% \verbatiminput though.
\usepackage{verbatim}

% Automatically remove leading whitespace in lstlisting
\usepackage{lstautogobble}

% PDF-compatible landscape mode.
% Makes PDF viewers show the page rotated by 90°.
\usepackage{pdflscape}

% Advanced tables
\usepackage{array}
\usepackage{tabularx}
\usepackage{longtable}

% Fancy tablerules
\usepackage{booktabs}

% Graphics
\usepackage{graphicx}

% Current time
\usepackage[useregional=numeric]{datetime2}

% Float barriers.
% Automatically add a FloatBarrier to each \section
\usepackage[section]{placeins}

% Custom header and footer
\usepackage{fancyhdr}

\usepackage{geometry}
\usepackage{layout}

% Math tools
\usepackage{mathtools}
% Math symbols
\usepackage{amsmath,amsfonts,amssymb}
\usepackage{amsthm}
% General symbols
\usepackage{stmaryrd}

\DeclarePairedDelimiter\abs{\lvert}{\rvert}

% Indistinguishable operator (three stacked tildes)
\newcommand*{\diffeo}{% 
  \mathrel{\vcenter{\offinterlineskip
  \hbox{$\sim$}\vskip-.35ex\hbox{$\sim$}\vskip-.35ex\hbox{$\sim$}}}}

% Bullet point
\newcommand{\tabitem}{~~\llap{\textbullet}~~}

\pagestyle{plain}
% \fancyhf{}
% \lhead{}
% \lfoot{}
% \rfoot{}
% 
% Source code & highlighting
\usepackage{listings}

% SI units
\usepackage[binary-units=true]{siunitx}
\DeclareSIUnit\cycles{cycles}

% Convenience commands
\newcommand{\mailsubject}{41106 - Cryptography Protocols - Series 1}
\newcommand{\maillink}[1]{\href{mailto:#1?subject=\mailsubject}
                               {#1}}

% Should use this command wherever the print date is mentioned.
\newcommand{\printdate}{\today}

\subject{41106 - Cryptographic Protocols}
\title{Series 1}

\author{Michael Senn \maillink{michael.senn@students.unibe.ch} - 16-126-880}

\date{\printdate}

% Needs to be the last command in the preamble, for one reason or
% another. 
\usepackage{hyperref}


\begin{document}
\maketitle


\setcounter{chapter}{0}

\chapter{Series 1}

\section{Calculation in finite fields}

\subsection{$r(X) = 3X^3 + 2X^2 + X \in GF(5)[X]$}

Given $GF(5)$ has prime order, we associate it with the ring of integers modulo
$5$, with the usual definitions for addition and multiplication.

As such evaluation of $r(X)$ at $X = 2$ is as follows:
\begin{align*}
		3X^3 + 2X^2 + X & \equiv 3 \cdot 2^3 + 2 \cdot 2^2 + 2 \\
						& \equiv 3 \cdot 3 + 2 \cdot 4 + 2 \\
						& \equiv 4 + 3 + 2 \\
						& \equiv 4 \pmod 5
\end{align*}

\subsection{$s(X) = (1 + \alpha) \cdot X^3 + \alpha \cdot X^2 + X \in GF(4)[X]$}

Given the provided addition and multiplication tables of $GF(4)$, evaluation of
$s(X)$ at $X = \alpha$ is as follows:

\begin{align*}
		(1 + \alpha) \cdot X^3 + \alpha \cdot X^2 + X & = (1 + \alpha) \cdot \alpha^3 + \alpha \cdot \alpha^2 + \alpha \\
													  & = (1 + \alpha) \cdot \alpha \cdot (1 + \alpha) + \alpha \cdot (1 + \alpha) + \alpha) \\
													  & = (1 + \alpha) \cdot 1 + 1 + \alpha \\
													  & = (1 + \alpha) + 1 + \alpha \\
													  & = \alpha + \alpha \\
													  & = 0
\end{align*}


\section{Trivial functions for secure computations}

\section{Non-trivial functions and embedded OR}


\end{document}
