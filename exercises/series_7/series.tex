\documentclass[a4paper]{scrreprt}

% Uncomment to optimize for double-sided printing.
% \KOMAoptions{twoside}

% Set binding correction manually, if known.
% \KOMAoptions{BCOR=2cm}

% Localization options
\usepackage[english]{babel}
\usepackage[T1]{fontenc}
\usepackage[utf8]{inputenc}

% Quotations
\usepackage{dirtytalk}

% Floats
\usepackage{float}

\usepackage{numbertabbing}

% Enhanced verbatim sections. We're mainly interested in
% \verbatiminput though.
\usepackage{verbatim}

% Automatically remove leading whitespace in lstlisting
\usepackage{lstautogobble}

% PDF-compatible landscape mode.
% Makes PDF viewers show the page rotated by 90°.
\usepackage{pdflscape}

% Advanced tables
\usepackage{array}
\usepackage{tabularx}
\usepackage{longtable}

% Fancy tablerules
\usepackage{booktabs}

% Graphics
\usepackage{graphicx}

% Current time
\usepackage[useregional=numeric]{datetime2}

% Float barriers.
% Automatically add a FloatBarrier to each \section
\usepackage[section]{placeins}

% Custom header and footer
\usepackage{fancyhdr}

\usepackage{geometry}
\usepackage{layout}

% Math tools
\usepackage{mathtools}
% Math symbols
\usepackage{amsmath,amsfonts,amssymb}
\usepackage{amsthm}
% General symbols
\usepackage{stmaryrd}

% Utilities for quotations
\usepackage{csquotes}

% Bibliography
\usepackage[
  style=alphabetic,
  backend=biber, % Default backend, just listed for completness
  sorting=ynt % Sort by year, name, title
]{biblatex}
\addbibresource{references.bib}

\DeclarePairedDelimiter\abs{\lvert}{\rvert}
\DeclarePairedDelimiter\floor{\lfloor}{\rfloor}

% Bullet point
\newcommand{\tabitem}{~~\llap{\textbullet}~~}

\floatstyle{ruled}
\newfloat{algo}{htbp}{algo}
\floatname{algo}{Algorithm}
% For use in algorithms
\newcommand{\str}[1]{\textsc{#1}}
\newcommand{\var}[1]{\textit{#1}}
\newcommand{\op}[1]{\textsl{#1}}

\pagestyle{plain}
% \fancyhf{}
% \lhead{}
% \lfoot{}
% \rfoot{}
% 
% Source code & highlighting
\usepackage{listings}

% SI units
\usepackage[binary-units=true]{siunitx}
\DeclareSIUnit\cycles{cycles}

% Convenience commands
\newcommand{\mailsubject}{41106 - Cryptography Protocols - Series 7}
\newcommand{\maillink}[1]{\href{mailto:#1?subject=\mailsubject}
                               {#1}}

% Should use this command wherever the print date is mentioned.
\newcommand{\printdate}{\today}

\subject{41106 - Cryptographic Protocols}
\title{Series 7}

\author{Michael Senn \maillink{michael.senn@students.unibe.ch} - 16-126-880}

\date{\printdate}

% Needs to be the last command in the preamble, for one reason or
% another. 
\usepackage{hyperref}

\begin{document}
\maketitle


\setcounter{chapter}{6}

\chapter{Series 7}

Let $|$ denote the concatenation of bit strings.

\section{Oblivious transfer from private set intersection}

Consider the following $\binom{2}{1}-\text{OT}$ protocol based on PSI.  Assume
a slightly modified $PSI$ protocol where $R$ learns of the set intersection
first, but unlike in the lecture does not communicate it to $S$. This can
easily be achieved as $S$ cannot determine the set intersection without the
help of $R$.

\begin{algo}
  \vbox{
    \small
    \begin{numbertabbing}
      xxxxxxxxxxxxxxxxxx\=xxxxxxxxxxxxx\=xxxxxxxx\=xxxxxxxxxxxxx\=MMMMMMMMMMMMMMMMMMM\=\kill
      $S(x_0, x_1)$ \> \> \> \> $R(b)$ \\
      $X = \{x_0 | 0, x_1 | 1\}$ \> \> \> \> $Y = \{0 | b, 1 | b\}$ \\
      \> $\xrightarrow{\mathmakebox[1cm]{X}}$ \> PSI \> $\xleftarrow{\mathmakebox[1cm]{Y}}$ \\
	  \> \> \> $\xrightarrow{\mathmakebox[1cm]{Z}}$ \\
	  \> \> \> \> Let $z$ be the first element of $Z$ \\
	  \> \> \> \> $z_0 | z_1 := z$ \\
      \> \> \> \> \textbf{return} $z_0$
    \end{numbertabbing}
  }
  \caption{2-1 OT using PSI}
  \label{alg:ot_from_psi}
\end{algo}

\subsection{Correctness}

From the definition of $X$ and $Y$ it directly follows that the result of the
private set intersection is $Z = X \cap Y = \{x_b | b\}$. As such it only
contains one element, and that element's first bit is indeed $x_b$.

\subsection{Security for S}

By the security property of PSI, a semi-honest $R$ learns only $\{x_b | b\}$,
and as such does not learn anything about $x_{1 - b}$. Clearly a malicious $R$
could submit $Y = \{00, 01, 10, 11\}$ and learn both values. In such a model,
additional security would have to be provided via e.g. a zero-knowledge proof.

\subsection{Security for R}

By the security property of our slightly modified PSI system, $S$ learns
nothing at all about $b$.

\section{Private set intersection from additively homomorphic encryption}

The following solution is based on a PSI algorithm for semi-honest adversaries
presented by Freedman, Nissim and Pinkas.  \autocite{efficientPSIPolynomials}

Let $A$ be a party with private input $X$ of order $n$, $B$ a party with
private input $Y$. Let $y \in Y$.  Consider the $n$-th degree polynomial $P(y)
= \prod_{x \in X}(x - y) = \sum_{i=0}^{n} a_i \cdot y^i$ over a finite field
$GF(q)$ of prime order. Note how by its constructions its roots are all the
elements of $X$, that is $P(y) = 0 \Leftrightarrow y \in X$.

Consider now the additively homomorphic ElGamal cryptosystem with public key
$pk$. Note how, in addition to allowing addition of ciphertexts, it also
permits multiplication of a ciphertext by a plaintext constant. Let $(R, C) =
(g^r, g^m \cdot pk^r)$ be an encryption of $m$. Then, $(R^c, C^c) = (g^{rc},
g^{mc} \cdot pk^{rc})$ is a valid encryption of $c \cdot m$.

\subsection{Set membership}

The following protocol allows $A$ to learn whether $P(y) = 0 \Leftrightarrow y
\in X$, without $B$ learning it as well.

\begin{algo}
  \vbox{
    \small
    \begin{numbertabbing}
			xx\=xx\=xxxxxxxxxxxxxxxxxxxx\=xxxxxxxxxxxx\=xx\=xx\=MMMMMMMMMMMMMMMMMMM\=\kill
			$A(X)$\>  \>                        \>            \> $B(y)$ \\
			// Encrypt coefficients of $P(y)$ \\
			\textbf{For} $i = 0$ to $n$: \\
			\> $c_i := \operatorname{AM-Enc}(pk, a_i)$ \> \> $\xrightarrow{\mathmakebox[1.5cm]{(c_0, \dots, c_n)}}$ \> $r \xleftarrow{R} GF(q)$ \\
			\> \> \> \> \textbf{For} $i = 0$ to $n$: \\
			\> \> \> \> \> $c'_i := (c_i)^{y^i}$ \\
			\> \> \> \> $c := \prod c'_i$ // $P(y)$ \\
			\> \> \> \> $c_{\text{blind}} := c^r$ // $r \cdot P(y)$ \\
			$m := \operatorname{AM-Dec}(sk, c_y)$ \> \> \> $\xleftarrow{\mathmakebox[1.5cm]{c_y}}$ \> $c_y := c_{\text{blind}} \cdot \operatorname{AM-Enc}(pk, y)$ // $r \cdot P(y) + y$ \\
			\textbf{Return} $m \in X$
	\end{numbertabbing}
  }
  \caption{Private set membership from additively homomorphic encryption}
  \label{alg:set_membership}
\end{algo}

\subsubsection{Correctness}

Note first how $c$ is an encryption of $P(y)$ due to the properties of the
additively homomorphic ElGamal cryptosystem. Then, $c_{\text{blind}}$ is an
encryption of $r \cdot P(y)$ and $c_y$ of $r \cdot P(y) + y$.

As established, $y \in X \Leftrightarrow P(y) = 0$. As $r \neq 0$ and $q$
prime, $P(y) = 0 \Leftrightarrow r \cdot P(y) + y = y$.

If $y \not \in X$ then $r \cdot P(y)$ is going to be a random value in $GF(q)$,
which might not be decipherable by $\operatorname{AM-Dec}$.  Being able to
decipher it is not required however, as it is sufficient to establish that it
is not a valid encryption of any $x \in X$ to conclude that $y \not \in X$.

Note that due to the blinding factor $r$ this is only a probabilistic
assertion. Specifically it is possible that there exists $x \in X$, $y \in Y$,
$x \neq y$, $r \in GF(q)$ such that $r \cdot P(y) + y \equiv x \pmod{q}$.

The chance of such a false positive happening are $\frac{k}{q}$, with $k$ the
size of the message space, $q$ the size of the finite field. With a properly
sized finite field this will be negligible.

\subsubsection{Security for A}

As $B$ only learns of encryptions of the polynomial's coefficents, $B$ learns
nothing about the secret input $X$ of $A$ due to the security of the ElGamal
cryptosystem. As $B$ further calculates $c_y$ using an additively homomorphic
cryptosystem it does also not learn whether $y \in X$, as it cannot distinguish
an encryption of $y$ from an encryption of a random value $k \in GF(q)$.

\subsubsection{Security for B}

Note how the blinding factor $r$ ensures that $r \cdot P(y) + y$ is a random
value if $y \not \in X$, as without it $A$ would be able to learn $y$ by using
the polynomial $G(y) = P(y) + y$, at which point $G^{-1}(P(y) + y) = y$.

As such, $A$ learns at most the value $k = r \cdot P(y) + y = y \Leftrightarrow
y \in X$.

\subsection{Private set intersection}

To extend this to set intersection, $B$ will simply execute its part for every
$y \in Y$, sending $c_{y_1} \ldots c_{y_m}$ where $m = \abs{Y}$. $A$ then
checks which of the received ciphertexts are valid encryptions of elements $x
\in X$, the union of which is equal to $X \cap Y$.

\begin{algo}
  \vbox{
    \small
    \begin{numbertabbing}
			xx\=xx\=xxxxxxxxxxxxxxxxxxxx\=xxxxxxxxxxxx\=xx\=xx\=MMMMMMMMMMMMMMMMMMM\=\kill
			$A(X)$\>  \>                        \>            \> $B(y)$ \\
			// Let $c_i$ be encryped coefficients of $P(y)$ as before \\
			\> \> \> $\xrightarrow{\mathmakebox[1.5cm]{(c_0, \dots, c_n)}}$ \\
			\> \> \> \> $r \xleftarrow{R} GF(q)$ \\
			\> \> \> \> \textbf{For} $i = 0$ to $m$: \\
			\> \> \> \> \> // Let $c_{y_i}$ be encryption of $r \cdot P(y_i) + y_i$ as before \\
			$C_y = \bigcup c_{y_i}$ \> \> \> $\xleftarrow{\mathmakebox[1.5cm]{(c_{y_1}, \ldots, c_{y_m}})}$ \\
			$Z = \{\}$ \\
			\textbf{For} $c_y \in C_y$: \\
			\> $m := \operatorname{AM-Dec}(sk, c_{y})$ \\
			\> \textbf{If} $m \in X$: \\
			\> \> $Z = Z \cup \{m\}$ \\
			\textbf{Return} $Z$
	\end{numbertabbing}
  }
  \caption{Private set intersection from additively homomorphic encryption}
  \label{alg:set_intersection}
\end{algo}

\subsubsection{Correctness}

Correctness of this private set intersection algorithm follows directly from
correctness of the set membership algorithm discussed earlier. Specifically:

\begin{align*}
		\forall\ y \in Y: y \in X \cap Y & \Leftrightarrow P(y) = 0 \\
				& \Leftrightarrow r \cdot P(y) + y = y \\
				& \Leftrightarrow \operatorname{AM-Dec}(sk, c_y) = y \in X
\end{align*}

Again the caveat about the potential for false positives applies.

\subsubsection{Security for A}

Security for $A$ is equal to security in the set membership algorithm, as the
exact same information is sent to $B$.

\subsubsection{Security for B}

Security for $B$ follows from security in the set membership algorithm, applied
to each $c_{y_i}$ it sends.

\section{Secure two-party AND using oblivious transfer}

Note first that for $x = 1$ the binary AND $z = x \land y$ clearly implies
$y = z$. Similarly $y = 1 \Rightarrow x = z$. As such an adversary submitting
$k = 1$ will always learn the secret value of the other party due to the nature
of the binary operation, and security is only relevant for $k = 0$.

Consider now the following protocol to calculate the binary AND $z = x \land y$
of two bits.

\begin{algo}
  \vbox{
    \small
    \begin{numbertabbing}
      xxxxxxxxxxxxxxxxxx\=xxxxxxxxxxxxx\=xxxxxxxx\=xxxxxxxxxxxxx\=MMMMMMMMMMMMMMMMMMM\=\kill
      $A(x)$ \> \> \> \> $B(y)$ \\
	  $x_0 = 0$ \\
	  $x_1 = x$ \\
      \> $\xrightarrow{\mathmakebox[1cm]{x_0, x_1}}$ \> OT \> $\xleftarrow{\mathmakebox[1cm]{y}}$ \\
	  \> \> \> $\xrightarrow{\mathmakebox[1cm]{x_y}}$ \\
	  \> $\xleftarrow{\mathmakebox[5cm]{x_y}}$ \\
	  \textbf{return} $x_y$ \> \> \> \> \textbf{return} $x_y$
    \end{numbertabbing}
  }
  \caption{Binary AND using 2-1 OT}
  \label{alg:and_from_ot}
\end{algo}

\subsection{Completness}

Note first how $y = 0 \Rightarrow x_y = x_0 = 0$. Further $x = 0 \Rightarrow
x_0 = x_1 = 0 \Rightarrow x_y = 0$. Lastly $x = y = 1 \Rightarrow x_y = x_1 =
1$. As such $x_y = x \land y$.

\subsection{Security for A}

As explained above, let $y = 0$. By the security property of $\binom{2}{1}$ OT,
B will learn nothing about $x_{1 - y} = x_1 = x$. As such nothing about the
private input $x$ is leaked.

\subsection{Security for B}

Similarly let $x = 0$. By the security property of $\binom{2}{1}$ OT, $A$ will
learn nothing about $y$ and knows only $x_0 = x_1 = 0$. Hence no information
about the private input $y$ is leaked.

\printbibliography

\end{document}
