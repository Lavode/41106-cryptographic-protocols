\documentclass[a4paper]{scrreprt}

% Uncomment to optimize for double-sided printing.
% \KOMAoptions{twoside}

% Set binding correction manually, if known.
% \KOMAoptions{BCOR=2cm}

% Localization options
\usepackage[english]{babel}
\usepackage[T1]{fontenc}
\usepackage[utf8]{inputenc}

% Quotations
\usepackage{dirtytalk}

% Floats
\usepackage{float}

\usepackage{numbertabbing}

% Enhanced verbatim sections. We're mainly interested in
% \verbatiminput though.
\usepackage{verbatim}

% Automatically remove leading whitespace in lstlisting
\usepackage{lstautogobble}

% PDF-compatible landscape mode.
% Makes PDF viewers show the page rotated by 90°.
\usepackage{pdflscape}

% Advanced tables
\usepackage{array}
\usepackage{tabularx}
\usepackage{longtable}

% Fancy tablerules
\usepackage{booktabs}

% Graphics
\usepackage{graphicx}

% Current time
\usepackage[useregional=numeric]{datetime2}

% Float barriers.
% Automatically add a FloatBarrier to each \section
\usepackage[section]{placeins}

% Custom header and footer
\usepackage{fancyhdr}

\usepackage{geometry}
\usepackage{layout}

% Math tools
\usepackage{mathtools}
% Math symbols
\usepackage{amsmath,amsfonts,amssymb}
\usepackage{amsthm}
% General symbols
\usepackage{stmaryrd}

\DeclarePairedDelimiter\abs{\lvert}{\rvert}
\DeclarePairedDelimiter\floor{\lfloor}{\rfloor}

% Indistinguishable operator (three stacked tildes)
\newcommand*{\diffeo}{% 
  \mathrel{\vcenter{\offinterlineskip
  \hbox{$\sim$}\vskip-.35ex\hbox{$\sim$}\vskip-.35ex\hbox{$\sim$}}}}

% Bullet point
\newcommand{\tabitem}{~~\llap{\textbullet}~~}

\floatstyle{ruled}
\newfloat{algo}{htbp}{algo}
\floatname{algo}{Algorithm}
% For use in algorithms
\newcommand{\str}[1]{\textsc{#1}}
\newcommand{\var}[1]{\textit{#1}}
\newcommand{\op}[1]{\textsl{#1}}

\pagestyle{plain}
% \fancyhf{}
% \lhead{}
% \lfoot{}
% \rfoot{}
% 
% Source code & highlighting
\usepackage{listings}

% SI units
\usepackage[binary-units=true]{siunitx}
\DeclareSIUnit\cycles{cycles}

% Convenience commands
\newcommand{\mailsubject}{41106 - Cryptography Protocols - Series 4}
\newcommand{\maillink}[1]{\href{mailto:#1?subject=\mailsubject}
                               {#1}}

% Should use this command wherever the print date is mentioned.
\newcommand{\printdate}{\today}

\subject{41106 - Cryptographic Protocols}
\title{Series 4}

\author{Michael Senn \maillink{michael.senn@students.unibe.ch} - 16-126-880}

\date{\printdate}

% Needs to be the last command in the preamble, for one reason or
% another. 
\usepackage{hyperref}

\begin{document}
\maketitle


\setcounter{chapter}{3}

\chapter{Series 4}

\section{Textbook ElGamal in Go}

Textbook ElGamal was implemented in Go, found in the attached source code.
Completness was experimentally tested by generating random numbers $x \in
\mathbb{Z}^*_p$, and verifying that $Dec(sk, Enc(pk, x)) = x\ \forall x$.

\section{Additively homomorphic ElGamal encryption}

Additively homomorphic ElGamal encryption as described in the lecture was
implemented too, again found in the attached source code. Completness was once
more experimentally tested. Random pairs of numbers $x, y < 10^4$ were
generated.  It was then verified that $\operatorname{AM-Dec}(sk,
\operatorname{AM-Enc}(pk, x) \otimes \operatorname{AM-Enc}(pk, y)) = x + y$,
where $\otimes$ is the component-wise multiplication  in $\mathbb{Z}_p^*$.

\subsection{Performance of AM-Dec}

Performance of the decryption operation was tested for two different keypairs,
with inputs in $[1, max]$ where $10^2 \leq max \leq 10^6$. Tests were done as
follows:

\begin{itemize}
		\item Two numbers $x, y$ in $[1, max / 2]$ were generated
		\item $(R_x, C_x) := \operatorname{AM-Enc}(pk, x)$, $(R_y, C_y) := \operatorname{AM-Enc}(pk, y)$
		\item $(R_z, C_z) := (R_x \cdot R_y, C_x \cdot C_y)$
		\item The duration it took for the decryption operation $z := \operatorname{AM-Dec}(sk, (R_z, C_z))$ to complete was measured
		\item The order of magnitude of $z$ as $\floor{\log_{10}{z}}$ and the measured duration were logged
\end{itemize}

The resulting data was aggregated by keypair and order of magnitude, and the
mean and standard deviation of the duration calculated. Data is shown in table
\ref{tbl:performance}.

As decryption of big inputs takes exceedingly long, sample sizes for larger
inputs were chosen to be smaller.

The large standard deviations can be explained by one magnitude spanning a large
range - $9999$ and $1000$ both have magnitude $3$, yet decryption of the former
will require nearly 10 times as many computations as the later.

\begin{table}
		\begin{tabular}{llllll}
				\toprule
				$\abs{p}$ & $\abs{q}$ & Order of magnitude of $z$ & Avg duration & Std dev & Sample count \\
				\midrule
				1024 & 160 & 2 & \SI{7.6}{\ms}   & \SI{2.6}{\ms}   & 1000 \\
				1024 & 160 & 3 & \SI{126.5}{\ms} & \SI{46.5}{\ms}  & 1000 \\
				1024 & 160 & 4 & \SI{2.0}{\s}    & \SI{6.1}{\s}    & 200 \\
				1024 & 160 & 5 & \SI{27.2}{\s}   & \SI{72.8}{\s}   & 20 \\
				1024 & 160 & 6 & \SI{291.0}{\s}  & \SI{26.4}{\s}   & 5 \\
				2048 & 256 & 2 & \SI{22.4}{\ms}  & \SI{7.3}{\ms}   & 1000 \\
				2048 & 256 & 3 & \SI{314.2}{\ms} & \SI{97.8}{\ms}  & 1000 \\
				2048 & 256 & 4 & \SI{4.1}{\s}    & \SI{1.2}{\s}    & 200 \\
				2048 & 256 & 5 & \SI{57.1}{\s}   & \SI{15.3}{\s}   & 20 \\
				2048 & 256 & 6 & \SI{665.4}{\s}  & \SI{256.0}{\s}  & 5 \\
				\bottomrule
		\end{tabular}
		\caption{Performance of AM-Dec decrypting plaintexts $z$ of differnt dimensions}
		\label{tbl:performance}
\end{table}

Figure \ref{fig:performance} visualizes this data in a logarithmic plot. It
clearly shows that generally processing power grows exponentially with the
magnitude of the input, which is expected as for an input of magnitude $k$,
$O(10^k)$ possible values need to be tried.

The figure also shows that different key parameters simply cause a linear
difference in processing time, which also matches with the expectation that the
performance of efficient modular exponentiation is linear in the bit length of
its parameters.

\begin{figure}[h]
        \centering
		\includegraphics[width=\textwidth]{elgamal_performance}
		\caption{Performance of decryption in additively homomorphic ElGamal}
		\label{fig:performance}
\end{figure}

\subsection{Improving performance of additively homomorphic ElGamal cryptosystem}

One natural way to optimize performance of the additively homomorphic ElGamal
cryptosystem is to parallelize the brute-force search in the decryption
function. For any non-trivial message space this search quickly uses up the
vast majority of processing time.

Recall that this brute-force search has the goal of finding an $i$ such that
$g^i \equiv h \pmod{p}$, which is done by iterating through permissible values
of $i$. As the individual iterations of the loop are independent of each other,
requiring only read-only access to a shared value of $h$, this task can be
evenly divided between $n$ processors. This will speed up processing by a
factor of $n$.

% TODO Different optimization

\end{document}
