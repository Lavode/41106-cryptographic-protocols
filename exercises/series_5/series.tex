\documentclass[a4paper]{scrreprt}

% Uncomment to optimize for double-sided printing.
% \KOMAoptions{twoside}

% Set binding correction manually, if known.
% \KOMAoptions{BCOR=2cm}

% Localization options
\usepackage[english]{babel}
\usepackage[T1]{fontenc}
\usepackage[utf8]{inputenc}

% Quotations
\usepackage{dirtytalk}

% Floats
\usepackage{float}

\usepackage{numbertabbing}

% Enhanced verbatim sections. We're mainly interested in
% \verbatiminput though.
\usepackage{verbatim}

% Automatically remove leading whitespace in lstlisting
\usepackage{lstautogobble}

% PDF-compatible landscape mode.
% Makes PDF viewers show the page rotated by 90°.
\usepackage{pdflscape}

% Advanced tables
\usepackage{array}
\usepackage{tabularx}
\usepackage{longtable}

% Fancy tablerules
\usepackage{booktabs}

% Graphics
\usepackage{graphicx}

% Current time
\usepackage[useregional=numeric]{datetime2}

% Float barriers.
% Automatically add a FloatBarrier to each \section
\usepackage[section]{placeins}

% Custom header and footer
\usepackage{fancyhdr}

\usepackage{geometry}
\usepackage{layout}

% Math tools
\usepackage{mathtools}
% Math symbols
\usepackage{amsmath,amsfonts,amssymb}
\usepackage{amsthm}
% General symbols
\usepackage{stmaryrd}

\DeclarePairedDelimiter\abs{\lvert}{\rvert}
\DeclarePairedDelimiter\floor{\lfloor}{\rfloor}

% Indistinguishable operator (three stacked tildes)
\newcommand*{\diffeo}{% 
  \mathrel{\vcenter{\offinterlineskip
  \hbox{$\sim$}\vskip-.35ex\hbox{$\sim$}\vskip-.35ex\hbox{$\sim$}}}}

% Bullet point
\newcommand{\tabitem}{~~\llap{\textbullet}~~}

\floatstyle{ruled}
\newfloat{algo}{htbp}{algo}
\floatname{algo}{Algorithm}
% For use in algorithms
\newcommand{\str}[1]{\textsc{#1}}
\newcommand{\var}[1]{\textit{#1}}
\newcommand{\op}[1]{\textsl{#1}}

\pagestyle{plain}
% \fancyhf{}
% \lhead{}
% \lfoot{}
% \rfoot{}
% 
% Source code & highlighting
\usepackage{listings}

% SI units
\usepackage[binary-units=true]{siunitx}
\DeclareSIUnit\cycles{cycles}

% Convenience commands
\newcommand{\mailsubject}{41106 - Cryptography Protocols - Series 5}
\newcommand{\maillink}[1]{\href{mailto:#1?subject=\mailsubject}
                               {#1}}

% Should use this command wherever the print date is mentioned.
\newcommand{\printdate}{\today}

\subject{41106 - Cryptographic Protocols}
\title{Series 5}

\author{Michael Senn \maillink{michael.senn@students.unibe.ch} - 16-126-880}

\date{\printdate}

% Needs to be the last command in the preamble, for one reason or
% another. 
\usepackage{hyperref}

\begin{document}
\maketitle


\setcounter{chapter}{4}

\chapter{Series 5}

\section{Proving in zero-knowledge that a graph has a Hamiltonian cycle}

Consider the described zero-knowledge proof to show existance of a Hamiltonian
Cycle $C$ in a graph $G = (V, E)$.

\subsection{Completness}

Let $P$ and $V$ be honest.

In the case $b = 0$, $V$ verifies that all commitments have been opened
correctly and that $H = \pi(G)$. The former holds due to their choice by $P$,
the later holds because $P$ chose $H$ as $\pi(G)$.

In the case $b = 1$, $V$ verifies that $D$ is a Hammiltonian cycle in $G$. As
$C$ is a Hammiltonian cycle in $G$, $D = \pi(C)$, $H = \pi(G)$ this holds by
their construction.

As such if $P$ and $V$ are honest, $V$ accepts and the completness property
holds.

\subsection{Soundness}

Let $P$ be malicious, ie not knowing a cycle $C$ in $G$. In the commitment
phase $P$ has to decide whether to generate a valid permutation $H = \pi(G)$ of
$G$, or whether to generate an arbitrary Hammiltonian cycle $D'$ on a graph $H'
\neq \pi(G)$.

If $P$ submits a valid permutation $H = \pi(G)$ he will pass the case of $b =
0$ but fail the case of $b = 1$ as he does not know of a Hammiltonian cycle in
$G$, and so does not know of one in $H$ either.

If $P$ submits a Hammiltonian cycle $D'$ on a graph $H' \neq \pi(G)$ he will
pass the case of $b = 1$ but fail the case of $b = 0$ as the verifier can
trivially check that $H'$ is not a permutation of $G$.

As such a malicious prover $P$ has a chance of $\frac{1}{2}$ of convincing a
verifier $V$, fulfilling the soundness requirement.

\subsection{Zero-knowledge}

Let the verifier flip a random coin $b$.

If $b = 0$ he can generate random permutation $H = \pi(G)$ and commit to it as
well as to all pairs of vertices $(v, w) \in V$ the same way the prover would.
In this case the permutation $pi$ as well as all commitments will have the same
distribution as those of an honest prover would.

If $b = 1$ he can generate an arbitrary (appropriately-sized) Hammiltonian
cycle $D'$ on $H' \neq \pi(G)$, and commit on the edges of this cycle as an
honest prover would. Again the distribution of $D'$ and commitments is equal to
the one generated by an honest prover.

As such the verifier learns nothing from the proof except that the statement
holds and the zero-knowledge property holds.

\section{Proving knowledge of an RSA-inverse}

Consider the described zero-knowledge proof of knowledge to show knowledge of
an $e$-th root of $h$ modulo $N$ in the RSA cryptosystem.

\subsection{Completness}

Let $P$ and $V$ be honest. Then:

\begin{align*}
		s^e & \equiv (r \cdot w^c)^e && \text{definition of $s$} \\
			& \equiv r^e \cdot w^{ce} \\
			& \equiv t \cdot w^{ce} && \text{definition of $t$} \\
			& \equiv t \cdot (w^{e})^{c} \\
			& \equiv t \cdot h^{c} \pmod{N} && \text{definition of $h$}
\end{align*}

As such the first condition $s^e \equiv t \cdot h^c \pmod{N}$ holds. The
seconds condition $\operatorname{gcd}(t, N) = 1$ also holds for the vast
majority of $t \in \mathbb{Z_N}$, as $N$ has exactly two non-trivial prime
factors $p$ and $q$. This condition might serve as a way to not trust a prover
who was able to factor the RSA modulus $N$, although such a prover could simply
determine the private RSA exponent $d$, extract $w := h^d \mod N$ and then
proceed with the proof the way an honest prover would.

\subsection{Soundness}

Assume the verifier had the ability to rewind the prover to an earlier state,
leading to two tuples $(t, c, s), (t, c', s'), c \neq c'$ with equal commitment
but different challenge and responses. That is:

\begin{align*}
		t & = g^r \\
		s & = r \cdot w^c \\
		s' & = r \cdot w^{c'}
\end{align*}

Without loss of generality assume that $c > c'$, else swap the two tuples.
Recall that $c, c' \in \mathbb{Z}_e$ so $c - c' < e$. As $e$ is prime it
follows that $\operatorname{gcd}(c - c', e) = 1$.

It then follows from Bézout's identity that there exist coefficients $a, b$
such that:
\[
		a \cdot e + b \cdot (c - c') = 1
\]

Let $w' = \left(\frac{s}{s'}\right)^b \cdot h^a$. Then:

\begin{align*}
		(w')^e & \equiv \left(\frac{s}{s'}\right)^{be} \cdot h^{ae} \\
			   & \equiv \left(\frac{s^e}{(s')^e}\right)^{b} \cdot h^{ae} \\
			   & \equiv \left(\frac{t \cdot h^c}{t \cdot h^{c'}}\right)^{b} \cdot h^{ae} \\
			   & \equiv h^{b (c - c')} \cdot h^{ae} \\
			   & \equiv h \pmod{N}
\end{align*}

Hence we extracted $w' \equiv w \pmod{N}$, showing soundness.

\subsection{Zero-knowledge}

Choose $c' \xleftarrow{R} \mathbb{Z}_e$, $s' \xleftarrow{R} \mathbb{Z}_N$, $t' =
s^e \cdot (h^c)^{-1}$.

Clearly $c'$ has the same distribution as $c$ in the actual protocol as it is
chosen in the same way. $s$ in the actual protocol is a product of two factors,
both of which are uniformly from $\mathbb{Z}_N$ so $s$ itself is too. This is
equal to the choice of $s'$ from $\mathbb{Z}_N$ in the simulation. $t = s^e
\cdot (h^c)^{-1} \equiv r^e \pmod{N}$ finally has the same properties as $t$ in
the actual protocol.

As such, $(t', c', s')$ has the same distribution as $(t, c, s)$ and the
protocol fulfills the zero-knowledge property.

Hence The

\end{document}
