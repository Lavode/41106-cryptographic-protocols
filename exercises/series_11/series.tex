\documentclass[a4paper]{scrreprt}

% Uncomment to optimize for double-sided printing.
% \KOMAoptions{twoside}

% Set binding correction manually, if known.
% \KOMAoptions{BCOR=2cm}

% Localization options
\usepackage[english]{babel}
\usepackage[T1]{fontenc}
\usepackage[utf8]{inputenc}

% Quotations
\usepackage{dirtytalk}

% Floats
\usepackage{float}

\usepackage{numbertabbing}

% Enhanced verbatim sections. We're mainly interested in
% \verbatiminput though.
\usepackage{verbatim}

% Automatically remove leading whitespace in lstlisting
\usepackage{lstautogobble}

% PDF-compatible landscape mode.
% Makes PDF viewers show the page rotated by 90°.
\usepackage{pdflscape}

% Advanced tables
\usepackage{array}
\usepackage{tabularx}
\usepackage{longtable}

% Fancy tablerules
\usepackage{booktabs}

% Graphics
\usepackage{graphicx}

% Current time
\usepackage[useregional=numeric]{datetime2}

% Float barriers.
% Automatically add a FloatBarrier to each \section
\usepackage[section]{placeins}

% Custom header and footer
\usepackage{fancyhdr}

\usepackage{geometry}
\usepackage{layout}

% Math tools
\usepackage{mathtools}
% Math symbols
\usepackage{amsmath,amsfonts,amssymb}
\usepackage{amsthm}
% General symbols
\usepackage{stmaryrd}

% Utilities for quotations
\usepackage{csquotes}

% Bibliography
\usepackage[
  style=alphabetic,
  backend=biber, % Default backend, just listed for completness
  sorting=ynt % Sort by year, name, title
]{biblatex}
\addbibresource{references.bib}

\DeclarePairedDelimiter\abs{\lvert}{\rvert}
\DeclarePairedDelimiter\floor{\lfloor}{\rfloor}

% Bullet point
\newcommand{\tabitem}{~~\llap{\textbullet}~~}

\floatstyle{ruled}
\newfloat{algo}{htbp}{algo}
\floatname{algo}{Algorithm}
% For use in algorithms
\newcommand{\str}[1]{\textsc{#1}}
\newcommand{\var}[1]{\textit{#1}}
\newcommand{\op}[1]{\textsl{#1}}

\pagestyle{plain}
% \fancyhf{}
% \lhead{}
% \lfoot{}
% \rfoot{}
% 
% Source code & highlighting
\usepackage{listings}

% SI units
\usepackage[binary-units=true]{siunitx}
\DeclareSIUnit\cycles{cycles}

% Convenience commands
\newcommand{\mailsubject}{41106 - Cryptography Protocols - Series 11}
\newcommand{\maillink}[1]{\href{mailto:#1?subject=\mailsubject}
                               {#1}}

% Should use this command wherever the print date is mentioned.
\newcommand{\printdate}{\today}

\subject{41106 - Cryptographic Protocols}
\title{Series 11}

\author{Michael Senn \maillink{michael.senn@students.unibe.ch} - 16-126-880}

\date{\printdate}

% Needs to be the last command in the preamble, for one reason or
% another. 
\usepackage{hyperref}

\begin{document}
\maketitle


\setcounter{chapter}{10}

\chapter{Series 11}

\section{Arithmetic circuits}

Let $GF(q)$ be a finite field of prime order $q$, with $q > 2$. Associate
`True' with $1 \in GF(q)$ and `False' with $0 \in GF(q)$. Then, a binary
circuit $C_b$ can be transformed to an equivalent arithmetic circuit $C_a$ by
replacing any of its four gates --- $\operatorname{AND}(x, y)$,
$\operatorname{OR}(x, y)$, $\operatorname{XOR}(x, y)$, $\operatorname{NOT}(x)$
--- with the arithmetic operations defined below.

\subsection{Logical `AND'}

Let $\operatorname{AND}(x, y) := x \cdot y \pmod{2}$. Then
$\operatorname{AND}(x, y) = 1 \Leftrightarrow x = y = 1$.

\subsection{Logical `NOT'}

Let $\operatorname{NOT}(x) := (x + 1) \pmod{2}$. Then $\operatorname{NOT}(0) =
1$ and $\operatorname{NOT}(1) = 0$.

\subsection{Logical `XOR'}

Let $\operatorname{XOR}(x, y) := (x + y) \pmod{2}$. Then $\operatorname{XOR}(0,
0) = \operatorname{XOR}(1, 1) = 0$ and $\operatorname{XOR}(0, 1) =
\operatorname{XOR}(1, 0) = 1$.

\subsection{Logical `OR'}

Note that $x \lor y = \neg(\neg x \land \neg y)$. Hence $\operatorname{OR}(x,
y) := (x + 1 \pmod{2}) \cdot (y + 1 \pmod{2}) + 1 \pmod{2}$. Then
$\operatorname{OR}(0, 0) = 0$, and $\operatorname{OR}(0, 1) =
\operatorname{OR}(1, 0) = \operatorname{OR}(1, 1) = 1$.

\section{Multiplication gate with preprocessing}

Let $GF(q)$ be a finite field of prime order, with $x, y, z, w_j, w_k, m_j,
m_k, w_t$ as per the exercise. Then:

\begin{align*}
		m_j m_k + m_j y + m_k x + z & = (w_j - x) \cdot (w_k - y) + (w_j - x) \cdot y + (w_k - y) \cdot x + xy \\
									& = w_j w_k - w_j y - w_k x + xy + w_j y - xy + w_k x - xy + xy \\
									& = w_j w_k
\end{align*}

Hence $m_j m_k + m_j y + m_k x + z = w_j w_k$ as required. As such the product
$w_t = w_j w_k$ can be calculated from $m_j, m_k, y, x, z$. Consider now how
each of the summands can be calculated.

By their construction, $[w_j - x], [w_k - y]$ can be calculated locally. This
then allows reconstructing $m_j, m_k$ as mentioned in the exercise, from which
$m_j \cdot m_k$ can be calculated locally. Similarly as $x$ and $y$ are known
to each party, $m_j \cdot y$ and $m_k \cdot x$ can be calculated locally as
well. Lastly $z$ is known to all parties as $z = x + y$.

Now that all summands are known, the addition can be done locally as well. This
then yields, as described earlier, the multiplication $w_t = w_j \cdot w_k$.

\end{document}
