\documentclass[a4paper]{scrreprt}

% Uncomment to optimize for double-sided printing.
% \KOMAoptions{twoside}

% Set binding correction manually, if known.
% \KOMAoptions{BCOR=2cm}

% Localization options
\usepackage[english]{babel}
\usepackage[T1]{fontenc}
\usepackage[utf8]{inputenc}

% Quotations
\usepackage{dirtytalk}

% Floats
\usepackage{float}

\usepackage{numbertabbing}

% Enhanced verbatim sections. We're mainly interested in
% \verbatiminput though.
\usepackage{verbatim}

% Automatically remove leading whitespace in lstlisting
\usepackage{lstautogobble}

% PDF-compatible landscape mode.
% Makes PDF viewers show the page rotated by 90°.
\usepackage{pdflscape}

% Advanced tables
\usepackage{array}
\usepackage{tabularx}
\usepackage{longtable}

% Fancy tablerules
\usepackage{booktabs}

% Graphics
\usepackage{graphicx}

% Current time
\usepackage[useregional=numeric]{datetime2}

% Float barriers.
% Automatically add a FloatBarrier to each \section
\usepackage[section]{placeins}

% Custom header and footer
\usepackage{fancyhdr}

\usepackage{geometry}
\usepackage{layout}

% Math tools
\usepackage{mathtools}
% Math symbols
\usepackage{amsmath,amsfonts,amssymb}
\usepackage{amsthm}
% General symbols
\usepackage{stmaryrd}

\DeclarePairedDelimiter\abs{\lvert}{\rvert}
\DeclarePairedDelimiter\floor{\lfloor}{\rfloor}

% Indistinguishable operator (three stacked tildes)
\newcommand*{\diffeo}{% 
  \mathrel{\vcenter{\offinterlineskip
  \hbox{$\sim$}\vskip-.35ex\hbox{$\sim$}\vskip-.35ex\hbox{$\sim$}}}}

% Bullet point
\newcommand{\tabitem}{~~\llap{\textbullet}~~}

\floatstyle{ruled}
\newfloat{algo}{htbp}{algo}
\floatname{algo}{Algorithm}
% For use in algorithms
\newcommand{\str}[1]{\textsc{#1}}
\newcommand{\var}[1]{\textit{#1}}
\newcommand{\op}[1]{\textsl{#1}}

\pagestyle{plain}
% \fancyhf{}
% \lhead{}
% \lfoot{}
% \rfoot{}
% 
% Source code & highlighting
\usepackage{listings}

% SI units
\usepackage[binary-units=true]{siunitx}
\DeclareSIUnit\cycles{cycles}

% Convenience commands
\newcommand{\mailsubject}{41106 - Cryptography Protocols - Series 6}
\newcommand{\maillink}[1]{\href{mailto:#1?subject=\mailsubject}
                               {#1}}

% Should use this command wherever the print date is mentioned.
\newcommand{\printdate}{\today}

\subject{41106 - Cryptographic Protocols}
\title{Series 6}

\author{Michael Senn \maillink{michael.senn@students.unibe.ch} - 16-126-880}

\date{\printdate}

% Needs to be the last command in the preamble, for one reason or
% another. 
\usepackage{hyperref}

\begin{document}
\maketitle


\setcounter{chapter}{5}

\chapter{Series 6}

\section{Soundness error}

Assume $P$ is a cheating prover, $V$ an honest verifier.

\subsection{ZKP for graph isomorphism}

Assuming $V$ does not know a graph $G_1$ isomorph to $G_0$, $V$ can generate
$H$ such that it is either isomorph to $G_0$, or to $G_1$. As such $V$ has to
correctly guess which value of $b$ the prover will generate in order to
convince $V$ in one round. Assuming $b$ is a fair coin, the soundness error is
hence $2^{-1}$.

With $k$ sequential rounds, the soundness error can be reduced to $2^{-k}$.

\subsection{ZKPK for discrete log}

% In order to convince the verifier, $P$ must provide an $s$ such that $g^s \cdot
% y^c \equiv t$. As $t$ is chosen before any information is received from the
% verifier, it can be considered a random value not controlled by $P$.
% 
% Finding $g^s$ is easy as $G$ is a group, but finding $s$ from it is the DLP so
% not feasible under the assumptions put on the system.

\subsection{ZKPK for RSA inverse}

\section{Proof-of-knowledge protocol of a representation}

\subsection{Soundness}

Let $(t, c, (s_1, \ldots, s_n)), (t, c', (s'_1, \ldots, s'_n))$ be two accepting
transcripts. Then:

\begin{align*}
		(s_i - s'_i) \cdot (c' - c)^{-1} & = ((r_i - c \cdot \alpha_i) - (r_i - c' \cdot \alpha_i)) \cdot (c' - c)^{-1} \\
										 & = (c' \cdot \alpha_i - c \cdot \alpha_i) \cdot (c' - c)^{-1} \\
										 & = \alpha_i \cdot (c' - c) \cdot (c' - c)^{-1} \\
										 & = \alpha_i
\end{align*}

So we have a knowledge extractor for $\alpha_1, \ldots, \alpha_n$.

\subsection{Zero knowledge}

Pick $c$ and $s_i$ randomly. Let $t = \prod g_i^{s_i} \cdot y^c$. The
verification condition of the protocol holds by the choice of $t$. $c$ is
chosen the same way it is in the protocol so has the same distribution. $s_i$
in the actual protocol is the difference between two values one of which is
chosen randomly, so too has the same distribution as in our simulation. $t$
lastly in the actual protocol is the product of random values so also has a
distribution equal to the one of the simulation. Hence the proof has the
zero-knowledge property.

\section{Encrypting a vote}

Consider the additively homomorphic ElGamal cryptosystem with public key $y$.

\subsection{Proving knowledge of plaintext of additively-homomorphic ElGamal ciphertext}

Given $i \in \mathbb{Z}_q$, $(R, C) = \operatorname{AM-Enc}(y, i) = (g^r, g^i
\cdot y^r)$ consider the following protocol to prove knowledge of $i$ such that
$(R, C)$ is a valid encryption thereof. This is an adaptation of the `proof of
representation' given in the lecture.

Note also how, in the context of a voting scheme, $i$ is likely to only take on
one of polynomially many values. As such a simple Schnorr proof for knowledge
of the discrete logarithm of $g^i$ would be insecure, as an adversary could
retrieve $i$ by brute force.

\begin{algo}
  \vbox{
    \small
    \begin{numbertabbing}
      xxxxxxxxxxxxxxxxxxxx\=xxxxxxxxxxxxxxxxxxxx\=MMMMMMMMMMMMMMMMMMM\=\kill
      \textbf{Prover}$(r, i)$ \>\> \textbf{Verifier}$(R, C)$ \\
      $r_1, r_2 \xleftarrow{R} \mathbb{Z}_q$ \\
      $t = g^{r_1} \cdot y^{r_2}$ \> $\xrightarrow{t}$ \> \\
      \> $\xleftarrow{c}$ \> $c \xleftarrow{R} \mathbb{Z}_q$ \\
      $s_1 = r_1 - c \cdot i$ \> \> \\
      $s_2 = r_2 - c \cdot r$ \> \> \\
      \> $\xrightarrow{s_1, s_2}$ \> \\
      \> \> Verify $t = g^{s_1} \cdot y^{s_2} \cdot C^c$
    \end{numbertabbing}
  }
  \caption{ZKPK of plaintext of additively homomorphic ElGamal}
  \label{alg:zkpk_elgamal}
\end{algo}

\subsubsection{Completness}

Consider a honest prover and verifier. Then:
\begin{align*}
  t & = g^{r_1} \cdot y^{r_2} \\
    & = g^{r_1 - c i + c i} \cdot y^{r_2 - c r + c r} \\
    & = g^{s_1} \cdot g^{c i} \cdot y^{s_2} \cdot y^{c r} \\
    & = g^{s_1} \cdot y^{s_2} \cdot (g^i \cdot y^r)^c \\
    & = g^{s_1} \cdot y^{s_2} \cdot (C)^c \\
\end{align*}

\subsubsection{Soundness}

Let $(t, c, (s_1, s_2)), (t, c', (s_1', s_2'))$ be two accepting transcripts.
Then:
\begin{align*}
  (c' - c) \cdot (s_1 - s_1') & = (c' - c) \cdot ((r_1 - ci) - (r_1 - c' i)) \\
                              & = (c' - c) \cdot (c' i - ci) \\
                              & = i
\end{align*}

And:
\begin{align*}
  (c' - c) \cdot (s_2 - s_2') & = (c' - c) \cdot ((r_2 - cr) - (r_2 - c' r)) \\
                              & = (c' - c) \cdot (c' r - cr) \\
                              & = r
\end{align*}

Are extractors for $r$ and $i$.

\subsubsection{Zero-knowledge}

Pick $c$, $s_1$ and $s_2$ randomly. Let $t = g^{s_1} \cdot y^{s_2} \cdot C^c$.
$c$ has the same distribution as in the actual protocol as it is chosen the
same way. $s_i$ being randomly chosen makes it have the same distribution as in
the actual protocol, where it is the difference between a randomly chosen $r_i$
and another number. The simulated $t$ lastly fulfills the requirement for
completness, while also being uniformly from the message space as in the actual
protocol.

\subsection{Proving validity of vote}

To have the prover $P$ prove that its vote is a valid encryption of either $v =
0$ or $v = 1$, consider first the proof of equality given in the lecture, which
allows to prove that one knows an $x$ such that:

\begin{align*}
  \log_{g_1}(y_1) = x = \log_{g_2}(y_2)
\end{align*}

Where $g_1$, $g_2$ are two independent generators. This can be applied to the
additively homomorphic ElGamal cryptosystem with $g_1 = g$, $g_2 = y$, $y_1 =
R$, $y_2 = C / g^v$. $g$ and $y$ are independent to the prover as it does not
know the secret exponent $x$ of the cryptosystem. Given an encryption $(R, C) =
(g^r, g^v \cdot y^r)$:

\begin{align*}
  \log_{g}(R) = r = \log_{y}(C / g^v)
\end{align*}

We then combine two such proofs using the `proof of disjunction' (`OR proof')
given in the lecture, for the following protocol. In the notation below we
assume that $v = 0$, for $v = 1$ simply swap all roles. Any variable with a $'$
denotes a variable which was simulated, any variable without one which was
generated according to the specification of the proof of equality. $\hat{r}$
denotes the blinding factor of the proof of equality, to prevent conflict with
$r$ which is the blinding factor of ElGamal (and also the witness for which we
want to prove knowledge).

\begin{algo}
  \vbox{
    \small
    \begin{numbertabbing}
      xxxxxxxxxxxxxxxxxxxx\=xxxxxxxxxxxxxxxxxxxx\=MMMMMMMMMMMMMMMMMMM\=\kill
      \textbf{Prover}$(r, i)$ \>\> \textbf{Verifier}$(R, C)$ \\
      // Correct proof for v = 0 \\
      $\hat{r} \xleftarrow{R} \mathbb{Z}_q$ \\
      $t_1 = g^{\hat{r}}$ \\
      $t_2 = y^{\hat{r}}$ \\

      // Simulate proof for v = 1 \\
      $c' \xleftarrow{R} \mathbb{Z}_q$ \\
      $s' \xleftarrow{R} \mathbb{Z}_q$ \\
      $t'_1 = g^{s'} \cdot R^{c'}$ \\
      $t'_2 = y^{s'} \cdot (C / g^1)^{c'}$ \\
      \> $\xrightarrow{t_1, t_2, t_1', t_2'}$ \\
      \> $\xleftarrow{\tilde{c}}$ \> $\tilde{c} \xleftarrow{R} \mathbb{Z}_q$ \\
      $c = \tilde{c} + c'$ \\
      $s = \hat{r} - c \cdot r$ \\
      \> $\xrightarrow{s, c, s', c'}$ \\
      \> \> Verify that: \\
      \> \> $t_1 = g^s \cdot R^c$ \\
      \> \> $t_2 = y^s \cdot (C / g^0)^c$ \\
      \> \> $t'_1 = g^{s'} \cdot R^{c'}$ \\
      \> \> $t'_2 = y^{s'} \cdot (C / g^1)^{c'}$
    \end{numbertabbing}
  }
  \caption{Proving validity of vote in homomorphic ElGamal}
  \label{alg:valid_vote}
\end{algo}

Completness, soundness and the zero-knowledge property follow directly from the
respective properties of the utilized proof of equality.

\end{document}
