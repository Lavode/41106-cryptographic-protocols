\documentclass[a4paper]{scrreprt}

% Uncomment to optimize for double-sided printing.
% \KOMAoptions{twoside}

% Set binding correction manually, if known.
% \KOMAoptions{BCOR=2cm}

% Localization options
\usepackage[english]{babel}
\usepackage[T1]{fontenc}
\usepackage[utf8]{inputenc}

% Quotations
\usepackage{dirtytalk}

% Floats
\usepackage{float}

\usepackage{numbertabbing}

% Enhanced verbatim sections. We're mainly interested in
% \verbatiminput though.
\usepackage{verbatim}

% Automatically remove leading whitespace in lstlisting
\usepackage{lstautogobble}

% PDF-compatible landscape mode.
% Makes PDF viewers show the page rotated by 90°.
\usepackage{pdflscape}

% Advanced tables
\usepackage{array}
\usepackage{tabularx}
\usepackage{longtable}

% Fancy tablerules
\usepackage{booktabs}

% Graphics
\usepackage{graphicx}

% Current time
\usepackage[useregional=numeric]{datetime2}

% Float barriers.
% Automatically add a FloatBarrier to each \section
\usepackage[section]{placeins}

% Custom header and footer
\usepackage{fancyhdr}

\usepackage{geometry}
\usepackage{layout}

% Math tools
\usepackage{mathtools}
% Math symbols
\usepackage{amsmath,amsfonts,amssymb}
\usepackage{amsthm}
% General symbols
\usepackage{stmaryrd}

\DeclarePairedDelimiter\abs{\lvert}{\rvert}
\DeclarePairedDelimiter\floor{\lfloor}{\rfloor}

% Indistinguishable operator (three stacked tildes)
\newcommand*{\diffeo}{% 
  \mathrel{\vcenter{\offinterlineskip
  \hbox{$\sim$}\vskip-.35ex\hbox{$\sim$}\vskip-.35ex\hbox{$\sim$}}}}

% Bullet point
\newcommand{\tabitem}{~~\llap{\textbullet}~~}

\floatstyle{ruled}
\newfloat{algo}{htbp}{algo}
\floatname{algo}{Algorithm}
% For use in algorithms
\newcommand{\str}[1]{\textsc{#1}}
\newcommand{\var}[1]{\textit{#1}}
\newcommand{\op}[1]{\textsl{#1}}

\pagestyle{plain}
% \fancyhf{}
% \lhead{}
% \lfoot{}
% \rfoot{}
% 
% Source code & highlighting
\usepackage{listings}

% SI units
\usepackage[binary-units=true]{siunitx}
\DeclareSIUnit\cycles{cycles}

% Convenience commands
\newcommand{\mailsubject}{41106 - Cryptography Protocols - Series 6}
\newcommand{\maillink}[1]{\href{mailto:#1?subject=\mailsubject}
                               {#1}}

% Should use this command wherever the print date is mentioned.
\newcommand{\printdate}{\today}

\subject{41106 - Cryptographic Protocols}
\title{Series 6}

\author{Michael Senn \maillink{michael.senn@students.unibe.ch} - 16-126-880}

\date{\printdate}

% Needs to be the last command in the preamble, for one reason or
% another. 
\usepackage{hyperref}

\begin{document}
\maketitle


\setcounter{chapter}{5}

\chapter{Series 6}

\section{Soundness error}

Assume $P$ is a cheating prover, $V$ an honest verifier.

\subsection{ZKP for graph isomorphism}

Assuming $V$ does not know a graph $G_1$ isomorph to $G_0$, $V$ can generate
$H$ such that it is either isomorph to $G_0$, or to $G_1$. As such $V$ has to
correctly guess which value of $b$ the prover will generate in order to
convince $V$ in one round. Assuming $b$ is a fair coin, the soundness error is
hence $2^{-1}$.

With $k$ sequential rounds, the soundness error can be reduced to $2^{-k}$.

\subsection{ZKPK for discrete log}

% In order to convince the verifier, $P$ must provide an $s$ such that $g^s \cdot
% y^c \equiv t$. As $t$ is chosen before any information is received from the
% verifier, it can be considered a random value not controlled by $P$.
% 
% Finding $g^s$ is easy as $G$ is a group, but finding $s$ from it is the DLP so
% not feasible under the assumptions put on the system.

\subsection{ZKPK for RSA inverse}

\section{Proof-of-knowledge protocol of a representation}

\subsection{Soundness}

Let $(t, c, (s_1, \ldots, s_n)), (t, c', (s'_1, \ldots, s'_n))$ be two accepting
transcripts. Then:

\begin{align*}
		(s_i - s'_i) \cdot (c' - c)^{-1} & = ((r_i - c \cdot \alpha_i) - (r_i - c' \cdot \alpha_i)) \cdot (c' - c)^{-1} \\
										 & = (c' \cdot \alpha_i - c \cdot \alpha_i) \cdot (c' - c)^{-1} \\
										 & = \alpha_i \cdot (c' - c) \cdot (c' - c)^{-1} \\
										 & = \alpha_i
\end{align*}

So we have a knowledge extractor for $\alpha_1, \ldots, \alpha_n$.

\subsection{Zero knowledge}

Pick $c$ and $s_i$ randomly. Let $t = \prod g_i^{s_i} \cdot y^c$. The
verification condition of the protocol holds by the choice of $t$. $c$ is
chosen the same way it is in the protocol so has the same distribution. $s_i$
in the actual protocol is the difference between two values one of which is
chosen randomly, so too has the same distribution as in our simulation. $t$
lastly in the actual protocol is the product of random values so also has a
distribution equal to the one of the simulation. Hence the proof has the
zero-knowledge property.

\section{Encrypting a vote}


\end{document}
