\section{Basic techniques}

\subsection{Programs as circuits}

Idea: Every program on inputs $x_1, \ldots x_n$ computes function $f(x_1,
\ldots x_n)$ which can be represented by a turing machine or a circuit.

\subsubsection{Cook-Levin theorem}

Every problem decideable by a non-deterministic turing machine in polynomial
tmie can be formulated as the SAT of a polynomial-size circuit.

\subsection{Public-key cryptography}

\subsubsection{Discrete logarithms}

Given $G = <g>$, the discrete log of $y \in G$ is $i$ such that $g^i = y$.

\paragraph{DLP}

Given $y = g^x$ for $x \in \mathbb{Z}_q$, find $x$.

\paragraph{CDH}

Given $x = g^a, y = g^b$ for $a, b \in \mathbb{Z}_q$, compute $g^{a \cdot b}$.

\paragraph{DDH}

Given $x = g^a, y = g^b, z = g^c$, where $c$ is either $ab$ or a random value
in $\mathbb{Z}_q$, differentiate the two cases.

\subsubsection{Public-key encryption formalization}

\begin{description}
		\item[Completness] $Dec(sk, Enc(pk, m)) = m \forall m$
		\item[Security] An encryption of any $m$ is indistinguishable from a random
				element of the ciphertext space. For two messages $m_1, m_2$ no
				adversary can distinguish $Enc(pk, m_1)$ from $Enc(pk, m_2)$.
\end{description}

\subsubsection{RSA Correctness}

Uses Fermat's little theorem, given $p$ prime, $a$ integer, $a, p$ coprime:
\begin{align*}
		a^{p-1} \equiv 1 \mod p
\end{align*}

And statement from CRT:
\begin{align*}
		a \equiv b \mod pq \Leftrightarrow a \equiv b \mod p \land a \equiv b \mod q
\end{align*}

Further:
\begin{align*}
		\phi(N) = (p - 1)(q - 1) \Rightarrow (p - 1) | \phi(N) \land (q - 1) | \phi(N)
\end{align*}

Then either $m \equiv 0 \mod p$ in which case $m^{ed} \equiv 0 \mod p$
directly, or:
\begin{align*}
		m^{ed} \equiv m^{ed - 1} \cdot m \equiv m^{(p-1)^h} \cdot m \equiv 1^h \cdot m \equiv m \mod p
\end{align*}

Equivalently for $\mod q$.
