\section{Oblivious two-party protocols}

\begin{description}
		\item[OPRF] Oblivious pseud-random function
		\item[PSI] Private set intersection
		\item[OT] Oblivious transfer]
\end{description}

\subsection{OPRF}

Recall PRF: $F : \{0, 1\}^\lambda \times \{0, 1\}^k \rightarrow \{1,
0\}^{out}$. $F(k, x)$ should `look random' to any party that does not know $k$.
In practice, PRF implemented as $F(k, x) \coloneqq HMAC(k, x)$.

Oblivious case: Sender $S$ provides $k$, receiver $R$ provides $x$.

\subsubsection{Properties}

\begin{itemize}
		\item As regular PRF: $R$ cannot distinguish $L^{OPRF}_{prf-real}$ from
				$L^{OPRF}_{prf-rand}$.
		\item Oblivious: For any two $x, x'$, $S$ cannot distinguish which one
				$R$ used as input to OPRF.
\end{itemize}


\subsubsection{Construction of PRF}

Given $G = <g>, |G| = q$, $H : \{0, 1\}^{in} \rightarrow G$, $I : \{0, 1\}^{*}
\rightarrow \{0, 1\}^{out}$ two hash functions, we construct a PRF $F:
\mathbb{Z}_q \times \{0, 1\}^{in} \rightarrow \{0, 1\}^{out}$ as follows:

\[
		F(k, x) \coloneqq I(H(x)^k)
\]

This is a PRF assuming hardness of CDH in the R.O.M.

\subsubsection{Oblivious evaluation of PRF $F$}

\begin{algorithm}
		\caption{Oblivious evaluation of PRF}

		\begin{multicols}{2}
				\begin{algorithmic}[0]
						\State \textbf{S}
						\State
						\State
						\State $w \coloneqq r^k \rightarrow$
				\end{algorithmic}

				\columnbreak

				\begin{algorithmic}[0]
						\State \textbf{R}
						\State $r \rgets \mathbb{Z}_q$
						\State $\leftarrow v \coloneqq H(x)^r$ \Comment{in G}
						\State
						\State $y \coloneqq J(w^{r^{-1}})$
				\end{algorithmic}
		\end{multicols}
\end{algorithm}

Security for $S$ holds also with malicious $R$, because it only sees a group
element. Security for $R$ holds for semi-honest $S$, a malicious $S$ could send
a $\hat{w} \neq w$. To fix this: E.g. ZKP.

\subsection{PSI}

$A$ holds set $X$, $B$ holds set $Y$. Goal: Calculate $X \cap Y$ in oblivious
way. Assume sets are represented as bit strings of some finite length.

\subsubsection{Insecure protocol}

\begin{enumerate}
		\item $A$ calculates $h_x \coloneqq H(x) \forall x \in X$, $B$
				calculates $h_y \coloneqq H(y) \forall y \in Y$
		\item $A$ sends $H_A$ the set of all hashes to $B$. $B$ intersects with
				the set of all hashes $H_B$
		\item $B$ calculates $Z = X \cap Y$, by checking for which $y \in Y$
				there is a hash value in $H_A \cup H_B$.
\end{enumerate}

This is insecure, as $B$ is able to test any element $z$ whether $H(z)$ is
contained in $H_A$.

\subsubsection{PSI using OPRF}

\begin{enumerate}
		\item $A$ picks OPRF key $k$
		\item $A$, $B$ transfer the PRF values of $y \in Y$, $H_B$ ($B$, acting
				as the recipient, learns them)
		\item $A$ sends PRF values of $x \in X$, $H_A$ (can compute them locally)
		\item $B$ computes intersection $Z = H_A \cap H_B$
		\item $B$ sends those $y \in Y$ for which there is an entry in $Z$
\end{enumerate}

