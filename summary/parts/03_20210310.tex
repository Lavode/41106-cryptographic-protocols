\subsection{Digital signatures}

\subsubsection{RSA}

Naive implementation using $Sign(sk, m) = m^d \mod N$ insecure, as adversary
could pick arbitrary $\sigma*, m := \sigma*^e$, which would pass verification.
Instead use hashed RSA signatures, `signing' $H(m)$ instead of $m$ directly.

\section{Blind signatures}

Goal: Protocol to allow $A$ to get $B$'s signature on $m$ without $B$ learning
the message, nor being able to associate a later signature with this action.
Compare: Signing envelope.

\subsection{Blind signatures for RSA}

Key generation as in RSA.

\begin{algorithm}
		\caption{Blind RSA signatures}
		\begin{algorithmic}[0]
				\Procedure{Sign}{$sk, m$}
				\State \textbf{A}
				\State $r \rgets \mathbb{Z}_N$
				\State $h' \coloneqq H(m) \cdot r^e \mod N$
				\State $h' \rightarrow B$
				\State \textbf{B}
				\State $s' \coloneqq h'^d \mod N$
				\State $A \leftarrow s'$
				\State \textbf{A}
				\State $s \coloneqq s' / r$
				\EndProcedure

				\Procedure{Verify}{$pk, m, s$}
				\State $s^e == H(m)$
				\EndProcedure
		\end{algorithmic}
\end{algorithm}

\subsection{Schnorr signatures (non-blind)}

Defined over $q$-order subgroup $G$ of $\mathbb{Z}^*_p$, with $p = m \cdot q +
1$.

\begin{algorithm}
		\caption{Schnorr signatures}
		\begin{algorithmic}[0]
				\Procedure{KeyGen}{}
				\State $x \rgets \mathbb{Z}_q$
				\State $Y \coloneqq g^x$
				\State \textbf{Return} $(Y, x)$
				\EndProcedure

				\Procedure{Sign}{$x, m$}
				\State $r \rgets \mathbb{Z}_q$
				\State $R \coloneqq g^r$
				\State $c \coloneqq H(m || R)$
				\State $s \coloneqq r - c \cdot x$
				\State \textbf{Return} $(c, s)$
				\EndProcedure

				\Procedure{Verify}{$Y, m, (c, s)$}
				\State \textbf{Return} $c == H(m || g^s \cdot Y^c)$
				\EndProcedure
		\end{algorithmic}
\end{algorithm}

\subsection{Blind schnorr signature scheme}

Key gen and verify same as in original.

\begin{algorithm}
		\caption{Blind Schnorr signatures}
		\begin{algorithmic}[0]
				\Procedure{Sign}{$x, m$}
				\State \textbf{B}
				\State $r \rgets \mathbb{Z}_q$
				\State $R \coloneqq g^r$
				\State $A \leftarrow R$

				\State \textbf{A}
				\State $\alpha, \beta \rgets \mathbb{Z}_q$
				\State $R' \coloneqq R \cdot g^{-\alpha} \cdot y^{-\beta}$
				\State $c' \coloneqq H(m || R')$
				\State $c \coloneqq c' + \beta$
				\State $c \rightarrow B$

				\State \textbf{B}
				\State $s \coloneqq r - c \cdot x$
				\State $A \leftarrow s$

				\State \textbf{A}
				\State $s' \coloneqq s - \alpha$
				\State \textbf{Return} $(c', s')$
				\EndProcedure
		\end{algorithmic}
\end{algorithm}

\subsection{Anonymous digital cash}

Given user $A$, shop $S$, bank $B$. Bank creates coins and stores balances, $A$
wants to exchange coins for services at $S$.

Goals:
\begin{itemize}
		\item If $A$ withdraws from $B$, then $B$ debits it from balance of $A$
		\item If $A$ transfers to $S$, then $B$ will credit coin to balance of $S$
		\item $B$ does not credit a coin to $S$ unless $B$ has issued the coin
				to some valid user $X$, and $X$ has transferred it to $S$
		\item If $B$ credits a coin to $Y$, then $B$ cannot link this coin to
				any withdrawal of a user.
\end{itemize}

\subsubsection{Protocol}

\begin{itemize}
		\item When $A$ wants to withdaw $u$ coins, it generates a random
				identifier $m$, blinds it as $m'$, and gets $B$ to issue a
				blind signature on $m'$. $B$ lowers $A$'s balance by $u$ and
				issues it a signature $\sigma'$ on $m'$. $A$ unblinds the
				signature to $\sigma$ and stores $(u, m, \sigma)$ in its
				wallet.
		\item To spend a coin, $A$ sends $(u, m, \sigma)$ to $S$. $S$ deposits
				it in the bank. If successful, it delivers goods or services to
				$A$.
		\item To deposit a coin, the bank verifies that $\sigma)$ is a valid
				signature for $m$. It also checks that $m$ was not yet in its
				list of already spent coins, and adds it to said list. If all
				is well, it increases the balance of $S$.
\end{itemize}
